\documentclass[DIN, pagenumber=false, fontsize=11pt, parskip=half]{scrartcl}

\usepackage{ngerman}
\usepackage[utf8]{inputenc}
\usepackage[T1]{fontenc}
\usepackage{textcomp}
\usepackage{amsmath}
\usepackage{amsfonts}
\usepackage{tikz}

% for matlab code
% bw = blackwhite - optimized for print, otherwise source is colored
\usepackage[framed,numbered,bw]{styles/exercise}
\usepackage{hyperref}

% for other code
%\usepackage{listings}

\setlength{\parindent}{0em}

% set section in CM
\setkomafont{section}{\normalfont\bfseries\Large}

\newcommand{\mytitle}[1]{{\noindent\Large\textbf{#1}}}
\newcommand{\mysection}[1]{\textbf{\section*{#1}}}
\newcommand{\mysubsection}[2]{\romannumeral #1) #2}


%===================================
\begin{document}

\noindent\textbf{Very Deep Learning} \hfill \textbf{Technische Universität Kaiserslautern} \\
WS 2022/23 \hfill Dr. Muhammad Zeshan Afzal \\

\mytitle{Exercise 5 - Semantic Segmentation and Object Detection }

\textbf{Deadline: 07.01.2023 \hfill Total Marks: 30}


\section*{Submission}

\begin{itemize}
    \item Submissions through OLAT. Only one group member needs to submit it.
    \item Your submission should contain a PDF with the solutions to the exercise questions (and a python notebook file) zipped together in a single file.
    \item Include the group number along with the names and matriculation numbers of all group members on the PDF.
    \item For the Jupyter notebook, please save them with the outputs of your code displayed.
\end{itemize}

%===================================
\mysection{5.1. Semantic Segmentation\hfill[3 + 3 + 4 = 10]}

\mysubsection{1}{ Applying transpose convolution on a $2\times2$ encoded feature map which needs to be upsampled to a $3\times3$ feature map. Kernel size is $2\times2$, stride is 1 with zero padding}

Given the following $2\times2$ input feature map:
\begin{table}[h!]
\centering
\begin{tabular}{|c|c|}
\hline
5 & 9 \\
\hline
0 & 3 \\ 
\hline
\end{tabular}
\end{table}

and the Final output feature map of size $3 \times 3$:
\begin{table}[h!]
\centering
\begin{tabular}{|c|c|c|}
\hline
 &  &  \\
 \hline
 &  &   \\
 \hline
 &  &   \\
\hline
\end{tabular}
\end{table}





% %===================================
\mysubsection{2}{Describe the concept of transpose convolution. Suppose 1D input of three elements [a,b,c] and filter with 4 elements [w, x, y, z], perform transpose convolution.}

\newpage
\mysubsection{3}{In this part, we will apply the most widely used upsampling techniques.}

Given the following $2\times2$ input feature map:
\begin{table}[h!]
\centering
\begin{tabular}{|c|c|}
\hline
8 & 16 \\
\hline
24 & 32 \\ 
\hline
\end{tabular}
\end{table}

and the Final output feature map of size $4 \times 4$:
\begin{table}[h!]
\centering
\begin{tabular}{|c|c|c|c|}
\hline
 &  &  &  \\
 \hline
 &  &  &   \\
 \hline
 &  &  &   \\ 
 \hline
 &  &  &   \\
\hline
\end{tabular}
\end{table}

Apply the following up-sampling techniques to convert the given $2\times2$ feature map into a $4\times4$ output.
\begin{enumerate}
    \item Nearest Neighbor
    \item Bi-Linear Interpolation
    \item Bed Of Nails
    \item Max-Unpooling : For Max-Unpooling, the saved indices are marked with \textit{x }in the following matrix:
 \begin{table}[h!]
\centering
\begin{tabular}{|c|c|c|c|}
\hline
 &  &  \textit{x }&  \\
 \hline
 &  \textit{x }&  &   \\
 \hline
 &  &  &\textit{x }\\ 
 \hline
 \textit{x }&  &  &   \\
\hline
\end{tabular}
\end{table}

\end{enumerate}




%===================================
\mysection{5.2. Object Detection \hfill[30]}

In this exercise, we will build a two-stage detector called Mask R-CNN. In order to set up everything and run the code on the Google Collab, do the following:
\begin{itemize}
    \item Download the dataset called \texttt{coco\_subset} from this \href{https://drive.google.com/drive/folders/107wKltnpX_hGD0YHloqM0PT0DdQLFgqK?usp=sharing}{Link}. It contains 2 directories (\texttt{annotations} and \texttt{train2017}).
    \item Create a folder, named \texttt{Exercise\_5} and upload the python notebook, \texttt{mrcnn} and coco\_subset directories in this folder. Now, your folder on google drive, Exercise\_5 should have two directories (\texttt{coco\_subset} and \texttt{mrcnn}) and one python notebook.
    \item Make sure that the folder \texttt{Exercise\_5} is created at the home location in Google drive. Otherwise, you have to modify the paths in the notebook accordingly.
\end{itemize}
 

Follow the instructions in the Jupyter notebook 
\texttt{Task\_5.2\_two\_stage\_detector.ipynb} to implement Mask R-CNN using PyTorch. Your task is to fill in the missing code annotated with \texttt{TODO} tags in the comments. The following are the required missing components in the notebook:
\begin{enumerate}
    \item Non Maximum Suppression
    \item Defining RPN
    \item RPN Forward Pass
    \item Classification and Box regression loss function 
    \item Mask loss function
    \item Defining R-CNN 
    \item R-CNN Forward Pass
\end{enumerate}


Good luck!

\end{document}